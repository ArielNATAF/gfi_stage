\documentclass[12pt,a4paper,twoside,titlepage]{article}
\usepackage{geometry}
\usepackage[]{microtype}
\usepackage[french,english]{babel}
\usepackage{enumitem}
\usepackage{setspace}
	\singlespacing
\usepackage[backend=biber]{biblatex}
\addbibresource{biblio.bib}
\usepackage[T1]{fontenc}
\usepackage{hyperref}
\usepackage{indentfirst}
\usepackage{listings}
% \usepackage[utf8]{inputenc}
\usepackage{color}
\usepackage{graphicx}
\usepackage{caption}
\usepackage{float}
\usepackage{titling}
\usepackage{pgfkeys}
\usepackage{array}
\usepackage{minted}
\usepackage[table,xcdraw]{xcolor}
\usepackage{amsmath}
\usepackage{hyperref}
\usepackage{xcolor}
\usepackage{titlesec}
\definecolor{RougeNanterre}{RGB}{227,0,27}
\usepackage[format=plain,
            font=it]{caption}
\hypersetup{
    colorlinks,
    linkcolor={black},
    citecolor={black},
    urlcolor={RougeNanterre}
}
\usepackage{fontspec}
  \setmainfont[Ligatures=TeX]{Noto Sans}
  \setsansfont[Ligatures=TeX]{Inconsolata}
  \setmonofont{Fira Mono}
% Define a new fontfamily for the subsubsection font
% Don't use \fontspec directly to change the font
\newfontfamily\subsubsectionfont[Color=RougeNanterre]{Times New Roman}
% Set formats for each heading level
\titleformat*{\section}{\Large\bfseries\sffamily\color{RougeNanterre}}
\titleformat*{\subsection}{\large\bfseries\sffamily\color{RougeNanterre}}
\titleformat*{\subsubsection}{\itshape\sffamily\color{RougeNanterre}}

\usepackage{csquotes}

\geometry{
    paper=a4paper,
    inner=3cm,
    outer=2.5cm,
    top=2.5cm,
    bottom=3.5cm
}

\usepackage{fancyhdr}
\fancyhead[RO,LE]{Ariel Nataf}
  \setlength{\headheight}{26pt}
  \renewcommand{\headrulewidth}{0.5pt}
  \renewcommand{\headrule}{\hbox to\headwidth{\color{red}\leaders\hrule height \headrulewidth\hfill}}
  \renewcommand{\footrulewidth}{0.5pt}
  \renewcommand{\footrule}{\hbox to\headwidth{\color{RougeNanterre}\leaders\hrule height \footrulewidth\hfill}}
\pagestyle{fancy}

\lstset{ %
  frame=none,
  backgroundcolor=\color{white},   % choose the background color; you must add \usepackage{color} or \usepackage{xcolor}
  basicstyle=\footnotesize\ttfamily,        % the size of the fonts that are used for the code
  breakatwhitespace=false,         % sets if automatic breaks should only happen at whitespace
  breaklines=true,                 % sets automatic line breaking
  captionpos=t,                    % sets the caption-position to bottom
  commentstyle=\color{mygreen},    % comment style
  deletekeywords={...},            % if you want to delete keywords from the given language
  escapeinside={\%*}{*)},          % if you want to add LaTeX within your code
  extendedchars=true,              % lets you use non-ASCII characters; for 8-bits encodings only, does not work with UTF-8
%  frame=single,                    % adds a frame around the code
  keepspaces=true,                 % keeps spaces in text, useful for keeping indentation of code (possibly needs columns=flexible)
  keywordstyle=\color{blue},       % keyword style
  language=,                 % the language of the code
  morekeywords={*,...},            % if you want to add more keywords to the set
  numbers=left,                    % where to put the line-numbers; possible values are (none, left, right)
  numbersep=5pt,                   % how far the line-numbers are from the code
  numberstyle=\tiny\color{mygray}, % the style that is used for the line-numbers
  rulecolor=\color{black},         % if not set, the frame-color may be changed on line-breaks within not-black text (e.g. comments (green here))
  showspaces=false,                % show spaces everywhere adding particular underscores; it overrides 'showstringspaces'
  showstringspaces=false,          % underline spaces within strings only
  showtabs=false,                  % show tabs within strings adding particular underscores
  stepnumber=1,                    % the step between two line-numbers. If it's 1, each line will be numbered
  stringstyle=\color{mymauve},     % string literal style
  tabsize=4,                       % sets default tabsize to 2 spaces
  aboveskip=3mm,
  belowskip=3mm,
}

\makeatletter
\newcommand\subTrois{\@startsection{paragraph}{4}{\z@}%
            {-2.5ex\@plus -1ex \@minus -.25ex}%
            {2.25ex \@plus .25ex}%
            {\sffamily\normalsize\itshape\color{RougeNanterre}}}
\newcommand\subQuatre{\@startsection{subparagraph}{5}{\z@}%
            {-2.5ex\@plus -1ex \@minus -.25ex}%
            {2.25ex \@plus .25ex}%
            {\sffamily\normalsize\itshape\color{RougeNanterre}}}
\newcommand\subCinq{\@startsection{subparagraph}{6}{\z@}%
            {-2.5ex\@plus -1ex \@minus -.25ex}%
            {2.25ex \@plus .25ex}%
            {\sffamily\normalsize\itshape\color{RougeNanterre}}}        
\makeatother

\newcommand{\coverpage}[1]{%
\begin{titlepage}
\newcommand{\HRule}{\rule{\linewidth}{0.5mm}} % Defines a new command for the horizontal lines, change thickness here

\center % Center everything on the page
 
%----------------------------------------------------------------------------------------
%	HEADING SECTIONS
%----------------------------------------------------------------------------------------

\textsc{\LARGE Rapport de stage de fin d'année}\\[0.3cm] % Name of your university/college
\textsc{\LARGE 1ère année de Master  }\\[0.3cm]
\textsc{\Large \color{RougeNanterre} Université Paris Nanterre }\\[0.3cm]
\textsc{\LARGE  à \color{RougeNanterre} GFI Informatique}\\[0.5cm] % Major heading such as course name
 % Minor heading such as course title

%--------------------------------------------------------------------------------
%	TITLE SECTION
%--------------------------------------------------------------------------------

\HRule \\[0.4cm]
{\sffamily \huge \bfseries Étude de caméras}\\[0.2cm]
{\sffamily \huge \bfseries et d'algorithmes de tracking}\\[0.1cm]
{\sffamily \huge \bfseries pour la Computer Vision sur OpenCV}\\[0.3cm]
% Title of your document
\HRule \\[1.5cm]

\vspace*{\fill}

%--------------------------------------------------------------------------------
%	AUTHOR SECTION
%--------------------------------------------------------------------------------

\begin{minipage}{0.4\textwidth}
\begin{flushleft} \large
Étudiant~:\\
\color{RougeNanterre}Ariel NATAF% Your name
\end{flushleft}
\end{minipage}
~
\begin{minipage}{0.4\textwidth}
\begin{flushright} \large
Maître de stage~:\\
\color{RougeNanterre}Jean-Paul MULLER% Supervisor's Name
\end{flushright}
\end{minipage}\\[1cm]

% If you don't want a supervisor, uncomment the two lines below and remove the section above
%\Large \emph{Author:}\\
%John \textsc{Smith}\\[3cm] % Your name

%--------------------------------------------------------------------------------
%	LOGO SECTION
%--------------------------------------------------------------------------------

\includegraphics{img/Logo_GFI_2011.jpg}\\[1cm]
\includegraphics{img/logo_Nanterre.jpg}\\[1cm]% Include a department/university logo - this will require the graphicx package

%--------------------------------------------------------------------------------
%	DATE SECTION
%--------------------------------------------------------------------------------

{\large avril-juillet \color{RougeNanterre}2018}\\[1cm] % Date, change the \today to a set date if you want to be precise
 
%--------------------------------------------------------------------------------

\vfill % Fill the rest of the page with whitespace
\end{titlepage}
}

% Custom arguments for /fig command
\pgfkeys{
 /fig/.is family, /fig,
 default/.style = 
  {scale = 1,
   angle = 0},
 scale/.estore in = \figScale,
 angle/.estore in = \figAngle
}
\newcommand{\fig}[2][]{%
	\pgfkeys{/fig, default, #1}%
	\begin{figure}[H]%
    \centering
    \includegraphics[angle=\figAngle,width=\figScale\textwidth]{#2}%
	\end{figure}%
}

\newcommand{\filename}[1]{%
	\texttt{#1}%
}

\newcommand{\vhdl}[1]{%
  \lstinputlisting[language=vhdl]{#1}
}

\newcommand*\paths[1]{\lstset{inputpath=#1}\graphicspath{#1}}

